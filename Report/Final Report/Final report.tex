%% LyX 2.0.8.1 created this file.  For more info, see http://www.lyx.org/.
%% Do not edit unless you really know what you are doing.
\documentclass[12pt,english]{article}
\renewcommand{\familydefault}{\rmdefault}
\usepackage[T1]{fontenc}
\usepackage[latin9]{inputenc}
\usepackage[letterpaper]{geometry}
\geometry{verbose,tmargin=2.5cm,bmargin=2.5cm,lmargin=2.5cm,rmargin=2.5cm}
\usepackage{units}
\usepackage{amstext}
\usepackage{amssymb}
\usepackage{graphicx}
\usepackage{setspace}
\setstretch{1.5}

\makeatletter

%%%%%%%%%%%%%%%%%%%%%%%%%%%%%% LyX specific LaTeX commands.
%% Because html converters don't know tabularnewline
\providecommand{\tabularnewline}{\\}
%% A simple dot to overcome graphicx limitations
\newcommand{\lyxdot}{.}


%%%%%%%%%%%%%%%%%%%%%%%%%%%%%% User specified LaTeX commands.
\usepackage[pagebackref]{hyperref}

\renewcommand\[{\begin{equation}}
\renewcommand\]{\end{equation}}

\makeatother

\usepackage{babel}
\begin{document}

\title{The Hall Effect and Properties of Semiconductors}
\maketitle
\begin{quotation}
\noindent \begin{center}
Emile JETZER
\par\end{center}

\noindent \begin{center}
Nicolas CHOUX, 
\par\end{center}

\noindent \begin{center}
David BOREL,
\par\end{center}

\noindent \begin{center}
McGill University
\par\end{center}
\end{quotation}
\pagebreak
\begin{abstract}
We investigated the Hall effect dependance on temperature in a germanium
crystal. We confirmed that the conductance in the sample follows a
power law in the intrinsic regime with $\chi_{2}^{2}=0.82$, at low
temperatures, and an exponential in the extrinsic regime, at high
temperatures. From the exponential fit we found band gap energy of
germanium to be $E_{g}=1.2(5)\times10^{-19}\,\mathrm{{J}}$. The Hall
coefficient follows the same trends, and this shows that the negative
charge carrier density of germanium decreases as temperature increases
in the intrinsic regime, and increases with temperature in the extrinsic
regime. We also verify that the magnetoresistance of germanium follows
a quadratic law with $\chi_{3}^{2}=0.902$.
\end{abstract}
\pagebreak{}
\tableofcontents
\pagebreak


\section{Introduction}

Some very interesting characteristics of crystals can be analyzed
experimentally. Such characteristics are the resistivity of a crystal
as a function of temperature, the hall voltage (this potential difference
created by the separation of charges in the crystal when a magnetic
field is applied on the crystal) as a function of temperature and
how the intensity of the magnetic field affects the resistance of
the crystal. In this experiment we will choose the germanium semiconductor
as our crystal to analyze. 

In order to analyze this characteristics we need to understand the
theory behind them. The electrons in the semiconductor have access
to different levels of energy. In semiconductors, these levels are
grouped in bands: The Conduction Band and the Valence Band. The seperation
between them is the energy gap. For insulators, electrons are in the
Valence Band, the energy gap between Conduction and Valence Band is
large and leaves no possibility for electrons to move around. In semiconductors,
the energy band gap is much smaller and the majority of the electrons
will be in the valence band but some will reach the conduction band
and be charge carriers. For carriers that are negatives with density
$n$ and charge $e$ for each charge, the current density $\textbf{J}$
relates to the mean velocity $\textbf{v}$ of the charge carriers
by : $\textbf{J}=en\textbf{v}$. It also relates to the conductivity
$\sigma=\nicefrac{1}{\rho}$ by the equation $\textbf{J}=\sigma\textbf{E}$.
The mean velocity $\textbf{v}$ and the mean free path $\lambda$
can be defined and we observe that $\lambda$ goes like $\nicefrac{1}{kT}$\cite{halleffect:online}.
Combining those results, for semiconductors, it can be shown that
the mean velocity $\textbf{v}\approx\frac{e\lambda\textbf{E}}{2\sqrt{3kTm}}$,
where $\lambda$ is the mean free path of the carriers, k is he Boltzmann
constant, T is the temperature of the sample and $m$ is the effective
carrier mass. Using this relation and the fact that $\lambda$ goes
like $\nicefrac{1}{kT}$ we find that the resistance $R$ is related
to the temperature $T$ with:
\begin{equation}
R\propto T^{3/2}\rightarrow G\propto T^{-3/2}\label{eq:Rpoly}
\end{equation}


This is the extrinsic conduction. However this equation is valid only
if the number of carriers remains constant. At higher temperatures,
we reach the limit where the carrier density varies with temperature.
In this case the conductivity follows Maxwell distribution:

\begin{equation}
R\propto e^{\nicefrac{a}{T}}\rightarrow G\propto e^{\nicefrac{Eg}{2kT}},\label{eq:Rexp}
\end{equation}


where $E_{g}$ is the band-gap energy. 

This is the intrinsic conduction. The conductance increases as electrons
are able to reach the conduction band.

Before introducing next characteristic we are interested in the experiment,
a common physical quantity, the mobility $\mu$ of the charge carriers,
is defined as $\mu=\frac{\textbf{v}}{\textbf{E}}=\frac{\sigma}{ne}$.
When a magnetic field is applied while current is driven through the
sample, a force is induced separating the carriers in such a way that
a potential difference is created across the sides of the sample.
This potential is called the Hall Voltage $V_{H}$. This effect can
also be quantified by the Hall coefficient defined to be

\begin{equation}
R_{H}=\frac{E_{y}}{J_{x}B}=\frac{V_{H}d}{IB}=\frac{1}{ne},\label{eq:coefficient}
\end{equation}
 where$E_{y}$ is the magnitude of the electric field created by the
charge pile-up , $J_{x}$is the current density in the sample, B is
the magnitude of the magnetic field and d is the thickness of the
sample (for a rectangle model perpendicular to the field). Another
way to quantify the effect is the Hall angle

\begin{equation}
\phi_{H}=\frac{V_{y}}{V_{x}}=\frac{E_{y}w}{E_{x}l}=\frac{w}{l}\mu_{H}B,\label{eq:mobility}
\end{equation}
 where $V_{x}$ is the Hall voltage drop across a width $w$ on the
sample, and $V_{y}$ is the voltage drop across a length $l$ on the
sample. $E_{x}$ and $E_{y}$ are the corresponding electric fields
and $\mu_{H}$ corresponds to the mobility. 

In a semiconductor the charge carriers can be negative carriers (carrier
density $n$) or positive carriers (carrier density $p$). The force
induced by the magnetic field, thus, push them in opposite direction
along a $y$direction according to the sign of the carriers. Realizing
that the net current must be zero and using the previous definition
we can obtain an equation for the Hall coefficient in terms of the
mobilities and the carriers densities: 
\begin{equation}
R_{H}=\frac{\mu_{+}^{2}p-\mu_{-}^{2}n}{e(\mu_{+}p-\mu_{-}n)}\label{eq:doping}
\end{equation}
These mobilities very with temperature. When $p>n$ we expect $R_{H}$
to change sign at some temperature (as $p$increases and $n$ decreases
with increasing temperature). Such semiconductors are said to be $\mbox{\ensuremath{p}-type}$.
Semiconductors which has $n>p$ are said to be $n\mbox{-type}$.

The resistance of the germanium sample is expected to change in the
presence of a magnetic field because the magnetic force shifts electrical
charges inside the sample. This affects the charge density, which
is not uniform anymore, and this, in turn, affects the resistance.
The difference between the resistivity in the presence of a magnetic
field, $\rho$, and without a field, $\rho_{0}$, is defined as $\Delta\rho\equiv\rho-\rho_{0}$.
For large values of $B$, $\Delta\rho$ is expected to have a quadratic
dependence on $B$\cite{magneto:online}

\begin{equation}
\Delta\rho\varpropto B^{2}.
\end{equation}


Then, $K(R-R_{0})=CB^{2}$, $R-R_{0\text{}}=CB^{2}$ ($K$ is absorbed
in $C$), and $R=CB^{2}+R_{0}$. Thus, the resistance $R$ is expected
to follow the form
\begin{equation}
R=C\cdotp B^{2}+D.\label{eq:magneto}
\end{equation}



\section{Function of conductance against temperature }

We can show that the resistance $R$ of a germanium crystal varies
with temperature $T$ as described in Eq.$\,$\ref{eq:Rpoly} at low
temperatures and \ref{eq:Rexp} at high temperatures. The setup used
is shown in Fig.$\,$\ref{fig:apparatus}. We use a germanium sample
connected to a thermocouple that measures the temperature of the crystal.
The sample is cooled down with liquid nitrogen to approximately $150\,\mathrm{{K}}$.
The temperature increases to room temperature and then a DC power
supply is used to power a heater connected to the sample to increase
the temperature up to $400\,\mathrm{K}$. During the experiment a
current source supplies a constant current of $1\mathrm{mA}$ to the
crystal. A digital multimeter (DMM) reads the potentials from the
sample holder, as well as the sample's temperature. These values are
displayed on the DMM's screen and are recorded on a computer.

\begin{figure}[h]
\centering\includegraphics[height=6cm]{\string"/home/ejetzer/Documents/hall/Graphs for final report/Apparatus\string".pdf}

\caption{\label{fig:apparatus}Schematic of the apparatus for the experiment.
On the left, the dials of each of the relevant parts, in the center,
the connections between parts and position of the sample, and on the
right the circuit embedded on the sample holder. The sample is connected
to the voltmeter, which feeds the potential measurements to the computer,
and supplies the current to the sample. The thermocouple labelled
is embedded into the sample holder, and connected to the voltmeter
via a cold-junction compensator.}
\end{figure}


The sample's conductance was plotted against temperature. At low temperature
when the conduction is mainly due to extrinsic process the number
of carriers is constant and we expect it to follow a power law $G\propto T^{-3/2}$
(see Eq.\inputencoding{latin1}{$\,$}\inputencoding{latin9}\ref{eq:Rpoly}).
At high temperature, the carrier density varies with temperature and
the conductance follows $G\propto e^{\nicefrac{b}{T}}$ (see Eq.\inputencoding{latin1}{$\,$}\inputencoding{latin9}\ref{eq:Rexp}).
We fitted the graph of conductance against temperature in the low
temperature region with a function of the form $a\times(x-x_{0})^{3/2}$
, with parameters $a$ and $x_{0}$. We also fitted the graph in a
high temperature region with a function of the form $a\times e^{bx}$,
with parameters are $a$ and $b$.

The $\chi_{2}^{2}$ for the power fit was found to be $\chi_{V_{5}}^{2}=0.82$
for $V_{5}$ and $\chi_{V_{6}}^{2}=0.85$ for $V_{6}$, indicating
that $G$ does follow a power law at low temperatures. The parameters
for the exponential fit were, for $V_{5}$ : $a=2.5(3)\times10^{-4}\,\Omega^{-1}$
, $b=4.54(5)\times10^{3}\,\mathrm{{K}}$ and the $\chi_{2}^{2}$ is
$\chi_{V_{5}}^{2}=1.07$ and $\chi_{V_{6}}^{2}=1.03$ for $V_{6}$,
and are plotted in Fig.$\,$\ref{fig:conductance}.

\begin{figure}[h]
\begin{centering}
\centering\includegraphics[height=6cm]{\string"/home/ejetzer/Documents/hall/Graphs for final report/GvsT\string".pdf}
\par\end{centering}

\caption{\label{fig:conductance}Fits at low and high temperature region for
graphs of conductance versus temperature when $V_{5}$ (blue) is applied
and when $V_{6}$ (green) is applied with the parameters of the exponential
fit $a=2.5(3)\times10^{-4}$ and $b=4.54(5)\times10^{3}$ for $V_{5}$
and $a=4.54(5)\times10^{3}$ and $b=4.48(5)\times10^{3}$ for $V_{6}$.
On the right is the way the apparatus is connected for this section
of the experiment.}
\end{figure}


At high temperature the conductance follows maxwell equation $G\propto e^{\frac{E_{g}}{2kT}}$,
where $E_{g}$ is the band gap energy. In order to find a value for
the band gap energy, we use the fit of the Conductance versus temperature
graph in the high temperature region where the fit is exponential.
Comparing the two equation we can see that $E_{g}=b\times2k=1.2(5)\times10^{-19}\,\mathrm{J}$,
where $b=4.54(5)\times10^{3}\,\mathrm{K}$.


\section{Hall coefficient as a function of Temperature}

The next step in the experiment is to determine the Hall coefficient
of the germanium sample as a function of temperature. This is done
by repeating the previous measurements, with the magnets powered to
supply uniform magnetic field of $500\,\mathrm{mT}$, as shown on
the right in Fig.$\,$\ref{fig:hallmobility}. The Hall coefficient
is computed from $V_{5}$ using Eq.$\,$\ref{eq:coefficient}, and
the Hall mobility from the manufacturer given sample characteristics
(Insert reference here) using Eq.$\,$\ref{eq:mobility}. As shown
on the plot, the Hall coefficient is decreasing in the extrinsic regime,
meaning the number of negative charge carriers is increasing with
temperature, as expected in Eq.$\,$\ref{eq:doping}. In the intrinsic
regime, the number of positive charge carriers is increasing.

\begin{figure}[h]
\centering\includegraphics[height=6cm]{\string"/home/ejetzer/Documents/hall/Graphs for final report/RHvsT\string".pdf}

\caption{\label{fig:hallmobility}Hall coefficient (blue) and mobility (green)
against temperature, and the way the apparatus is connected during
this section of the experiment. The magnets are on, and the Hall probe
is used to measure the magnetic field.}


\end{figure}



\section{Magneto-resistance against magnetic field intensity}

In this section, we seek to determine how the resistance of the germanium
sample is affected by the magnetic field. We put the sample inside
the magnet. There is a constant current of 1 mA passing through the
sample. We record the potential of the Hall probe $V_{probe}$, and
the potential $V{}_{5}$ (see Fig.$\,$\ref{fig:apparatus}). We varied
the magnetic field from 50 to $500\,\mathrm{mT}$. The resistance
$R$, we divide $V{}_{5}$ by the current of 1$\,$mA. The magnetic
field $B$ depends on $V_{probe}$ according $B=\gamma+\delta\cdotp V_{probe}$.
Our probe was calibrated prior to the experiment, and has parameters
$\gamma=9.62$ mT, and $\delta=1.8949$ T/V. The plot of the $R$
against $B$ is displayed in Figure \ref{fig:magneto-resistance}.

\begin{figure}[h]
\centering\includegraphics[height=6cm]{\string"/home/ejetzer/Documents/hall/Graphs for final report/Magnetoresistance\string".pdf}

\caption{\label{fig:magneto-resistance}The resistance $R$ ($\Omega$) against
the magnetic field $B$ ($10^{-1}$T) of the germanium sample. $R$
increases quadratically with $B$. The results of the fit are shown
in Tab.$\,$\ref{tab:magneto-resistance}.}
\end{figure}


We used a fit of the form $R=a(B-B_{0})^{2}+b$, where $B_{0}$ is
a constant. The parameter values and $\chi_{3}^{2}$ are written in
Table \ref{tab:magneto-resistance}. The $\chi_{3}^{2}$ is $0.902$,
thus we conclude that the resistance of germanium does indeed depend
on the magnetic field quadratically.

\begin{table}[h]
\centering%
\begin{tabular}{|c|c|}
\hline 
Parameter & Value\tabularnewline
\hline 
\hline 
$a$ ($\Omega$/T) & 9.2(9)\tabularnewline
\hline 
$B_{0}$ (T) & -0.28(5)\tabularnewline
\hline 
$b$ ($\Omega$) & 65.4(3)\tabularnewline
\hline 
$\chi^{2}$ & 0.902\tabularnewline
\hline 
\end{tabular}

\caption{\label{tab:magneto-resistance}The parameter values of the fit for
the magneto-resistance. The fit has the functional form $R=a(B-B_{0})^{2}+b$.}
\end{table}



\section{Potential error}

To determine the error on the measurements of potential, we take series
of measurements at 140, 300 and 400$\,$K and took the standard deviation
to obtain the uncertainty at low, room and high temperatures. The
errors are propagated by adding the product of the partial derivatives
and uncertainties of parameters in quadrature.


\paragraph{\pagebreak{}}

\bibliographystyle{IEEEtran}
\bibliography{MyBibliography}

\end{document}
